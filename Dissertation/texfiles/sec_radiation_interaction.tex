\section{Radiation Interaction with Matter}\label{radiation_basics}
Before delving into the methods of neutron spectrum unfolding, a basis must be set for how neutrons interact with matter. Being uncharged particles, neutrons cannot be affected by electric fields and cannot be as easily manipulated into interactions as charged particles can. The likelihood of neutrons interacting with matter depends solely on the cross-section of interaction \cite{lamarsh_baratta_2018}. The only options available are to use detection materials that have a high cross-section of interaction with neutrons or to use other materials with high cross-sections of interaction to cause the neutrons to slow down.

Neutrons can interact with matter in myriad ways. Because they are not charged particles, neutrons interact directly with the nucleus and see no effect due to the electron cloud around the nucleus \cite{lamarsh_baratta_2018}. The types of interactions include: elastic scattering, inelastic scattering, radiative capture, charged-particle reactions, neutron-producing reactions, and fission. \cite{lamarsh_baratta_2018}. Each type of interaction has a characteristic probability of occurring depending on the neutron energy. In other words, the cross-section depends on the type of interaction and energy of the incident neutron. An example graph of interaction cross-section as a function of neutron energy is shown in \textbf{Figure below}.

\textbf{get cross-section vs energy graph}

Most radiation encountered by daily radiation workers has energies up to 20 MeV \cite{lamarsh_baratta_2018}.