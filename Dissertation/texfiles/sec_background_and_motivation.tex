\section{Background and Motivation}\label{background}
The ability to accurately measure and calculate radiation dose to within certain limits is a standard set my national and international government organizations \cite{doe-std-1098}. On a regular basis, the capabilities of nuclear enterprises are tested against these standards through multi-organizational exercises \textbf{cite: either DOE O 420.1C, 10 CFR Part 830, or IER-538 CED4A Report}. The purpose of these exercises is to acquire detector responses through dosimeters that can be used to obtain dose information \textbf{10 CFR Part 830}. Acquiring dose information requires multiple steps from detector response to dose. First, the detector response needs to be unfolding to show the neutron spectrum. This process is not well defined and is currently under study to improve it. Second, that unfolded neutron spectrum must be converted to dose through reported energy-dependent conversion factors \cite{compendium_of_neutron_spectra}.

The study of neutron spectrum unfolding can be traced back to the 1950s \cite{poole_1952} with further work on developing specific neutron spectrum unfolding techniques in the 1960s \cite{habiger_1966}. The accurate unfolding and characterization of neutron spectra from detector responses in important in many fields of study and in practice. In the last 60-70 years, various methods have been developed to improve the accuracy and reliability of spectrum unfolding, including but not limited to: gathering information from proton recoil \cite{wu_guung_pei_yang_hwang_thomas_1999}, using nuclear reaction products \cite{dietz_matzke_sosaat_urbach_weyrauch_1993}, time-of-flight methods \cite{colonna_tagliente_1998}, threshold methods \cite{kuijpers_herzing_cloth_filges_hecker_1977}, and multiple sphere systems \cite{bramblett_ewing_bonner_1960}.

Radiation is prevalent in our lives, whether it originates from background radiation, cosmic rays, medical processes, or nuclear reactors. Understanding the effects of radiation is crucial for the ability to utilize it as well as ensure the least harm comes from it. The study of health physics is concerned with, among other things, the physical measurements of radiation and the relationship that arises between radiation exposure and biological damage \cite{johnson_cember_2017}. Additionally, the effects of radiation on non-biological materials is another major area of study \textbf{(need citation)}. To determine the effects of any kind of radiation that has interacted with a material, the energy of that radiation must be known \cite{johnson_cember_2017}. 

The unit of radiation energy is the electron volt (eV) \cite{johnson_cember_2017}. This unit is derived from the energy required to move an electron through one volt of potential difference. The electron volt is equivalent to $1.6 \cdot 10^{-19}$ Joules. Radiation can have a range of energy from meV ($10^{-3} eV$) to GeV ($10^9 eV$).