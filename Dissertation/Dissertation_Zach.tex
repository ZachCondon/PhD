%************************************************
%************************************************
%************************************************
% UNOFFICIAL OHIO STATE COLLEGE OF
% ENGINEERING MS/PHD/CANDIDACY TEMPLATE
%************************************************
%************************************************
% Template by Richard Vasques
% Last updated December 2019
%************************************************
%************************************************

% This is a template for my current and future students that will be writing their documents in LaTeX.
% This is an adaptation of the template by Swarnendu Biswas that can be found at https://github.com/swarnendubiswas/ohio-state-coe-dissertation-template.
% This template attempts to follow the rules prescribed by the Graduate School and College of Engineering, and is (hopefully) up to date with the graduate school requirements as of December 2019. 
% This adaptation is intended for students in my research group only, and is not officially supported by The Ohio State University, with no claims being made about its conformity with the current requirements.
% (In fact, this is a hastily put together template, as I do not have the time I
% wish to go deeply into the sty and cls files to make something nicer)

%************************************************
%************************************************

% This file (Dissertation-template) is the source file; that means, the one you need to compile.
% I recommend you name this file after yourself (e.g. Lastname_Dissertation.tex).

%************************************************
%************************************************

% Have the following files in the same folder:
%
% osudissert96-mods.sty
% osudissert96.cls
% This_File.tex (this current file, currently named Dissertation-template.tex, hopefully renamed by you)
% references.bib (this is your bibliography)

%************************************************
%************************************************
% Also, have the subfolders ``texfiles" and ``images".
% In ``images" you will place all figures and images.
% In ``texfiles" you will place all other tex files (duh) of your document,
% including for this template:
% preamble.tex 
% abstract.tex
% acknowl.tex
% vita.tex
% symbols.tex
% introduction.tex
% chapter2.tex
% conclusions.tex
% app1.tex
% and any other tex files you need 

%************************************************
%************************************************

% To change the dissertation to a Master's Thesis, use a documentclass option such as [masters] or [ms].
% To change to a PhD candidacy proposal, use a documentclass option [candidacy].
% The default option is phd.
% Also available are [osudraft] and [twoside].
% As a reminder, documentclass options are a comma-separated list, e.g. \documentclass[ms,osudraft]{osudissert96}

%\documentclass[ms]{osudissert96}
%\documentclass[candidacy]{osudissert96}
\documentclass{osudissert96}

%************************************************
%************************************************

% Put all package definitions in the file Preamble.tex:
\input{texfiles/preamble}
% I have populated the file Preamble.tex with the packages you will
% probably need.
% Feel free to add/change/remove packages according to your needs, as long
% as it works without breaking anything else. 

%************************************************
%************************************************

% It is better to break up the dissertation into multiple files (e.g.,
% one file per chapter, as well as separate files for the abstract,
% acknowledgments, vita, etc.).  These files are brought into the
% document using \include{} statements.  There will be times, however,
% when you don't want to print the ENTIRE dissertation.  You can limit
% what will actually be printed by using the \includeonly{} statement.
% This contains a list of the files you want printed.  Any file NOT
% listed will not be printed.  However, all page numbers, references,
% etc., will be preserved as though all the files were actually
% printed. For example, the line below would result only in chapters 
% "introduction" and "chapter2" being printed (if it were uncommented).

% \includeonly{introduction,chapter2}

%************************************************
%************************************************

% UPDATED TEXT (2010):
% In the newest format, titles should be title case everywhere.
\renewcommand\typesetChapterTitle[1]{#1}

%************************************************
%************************************************

\begin{document}

% First, declare the parts of your title page

%Your full name
\author{Firstname Midname Lastname}

%Title of your work
\title{Ohio State College of Engineering MS/PhD/Candidacy Dissertation Template}

% Your Degrees thus far, not including the one for this document. Ex: B.S., B.Sc., B.E., M.Sc., M.S., etc.
\authordegrees{B.S., M.S.}  

%Program name (TeX will add the words "Graduate Program in"
\unit{Nuclear Engineering}

% Committee members
\advisorname{Richard Vasques} %This should probably be me
\member{Someone 1, Co-Advisor} % Only use Co-Advisor if needed; otherwise just enter the name of the committee member. 
\member{Someone 2}
\member{Maybe Someone 3}

% The following creates the title page
\maketitle

%************************************************
%************************************************

% Next, EITHER a copyright or BLANK page.
%
%   The following creates a page used to copyright your dissertation
%
%   BACKGROUND: Even without this copyright page, your dissertation will
%               carry a common-law copyright. However, if your
%               dissertation ends up seeing wide distribution, your
%               common-law copyright is at risk of being expunged.
%               Adding this copyright page prevents that from happening.
%
%               There are NO DOWNSIDES to including a copyright page as
%               your document is automatically copyright by law anyway.
%               However, this copyright page is OPTIONAL. If you get rid
%               of it, uncomment the \blankpage that follows it so that
%               there is a blank page here. The graduate school requires
%               a page here that is either blank or carries the
%               copyright.
%
%   IMPORTANT NOTE: The graduate school requires either a copyright page
%                   here or a BLANK PAGE here. If you get rid of the
%                   copyright, uncomment the \blankpage that follows it.
%                   You should NOT have BOTH uncommented.
%

% If you get rid of \disscopyright, restore the \blankpage line after it
\disscopyright{}
%\blankpage

%************************************************
%************************************************

% Bring in abstract from separate file named ``abstract.tex''
 \begin{abstract}
	 %Abstract.

This is your abstract.
Fill it accordingly. 

\lipsum[1-3]


 \end{abstract}

%************************************************
%************************************************

%  Dedication goes after the abstract. Dedication is not needed (up to you), and
%should not be present in candidacy proposal. Refrain from adding a dedication
%page until after the defense.
\dedication{\emph{Dedicated to elevators for always lifting people up and being helpful on so many levels.}}

%************************************************
%************************************************

% UPDATED TEXT (2010):
%  The graduate school does not require an external abstract. If this
%  changes, follow the old instructions below.
%
% HISTORICAL TEXT (1996):
%  Uncomment the three lines below to generate the external abstract.  Two
%  copies of this must be turned in to the graduate school.  These lines can
%  be placed pretty much anywhere, since the page numbering should be
%  independent of the rest of the thesis
%

% \begin{externalabstract}
%   %Abstract.

This is your abstract.
Fill it accordingly. 

\lipsum[1-3]


% \end{externalabstract}

%************************************************
%************************************************

% If this is a PhD or MS document, include the following:
% Bring in Acknowledgment from separate file named ``acknowl.tex'' or some other name you defined
\begin{acknowledgements}
I thank my friends and family, without whom this work would have been completed two years earlier.

In reality, this is the only page of the dissertation of which the author has full control.
You can write anything you want here, and no one can tell you it is wrong (except if the margins don't line up!!!!).
\end{acknowledgements}

 % Do not include for a candidacy proposal, and 
%refrain from including it until after the defense.

%************************************************
%************************************************

% Bring in Vita from separate file named ``vita.tex'' or some other name you defined
\begin{vita}

\dateitem{August 2016}{B.S. in something, The Ohio State University}

\dateitem{Some date}{Some degree, Some place}

\dateitem{September 2016 to present}{Graduate Research Associate,\\The Ohio State University}

\begin{publist}

%% UPDATE FOR 2010:
%  Grad school only wants research publications, and it only wants those
%  research pubs that are actually published. Accepted or ``to appear''
%  publications don't count. If they look closely, they'll tell you to
%  remove any publications that aren't in print. Having said that, they
%  probably won't look that closely unless you put a really long list
%  here. You're tempting fate if you add instructional publications
%  though.

\researchpubs{}

\pubitem{F.M.~Lastname, J.~Doe, and R.~Vasques, ``Some Cool Title for a Paper," Journal Name, vol.~99, pp.~11-22, 2018.
}

% \instructpubs
%
% \pubitem{B.~Simpson, ed.,
% \newblock ``Lab notes for Cow Science 101'', 1909.}

\end{publist}

\begin{fieldsstudy}

% The \majorfield* uses the unit specified in the \unit command used
% earlier in your document.
 \majorfield*

%If you want to use different units, use the
% second form shown below:
%\majorfield{Computer Science and Engineering}
%\begin{studieslist}
%\studyitem{XX}{Prof.\ XX}
%\studyitem{YY}{Prof.\ YY}
%\studyitem{ZZ}{Prof.\ ZZ}
%\end{studieslist}

%% Note:  If there were only one extra field of study, the list
  %% would best be done using the following command:
%%
%%  \onestudy{Only Topic}{Only Professor}
%%

\end{fieldsstudy}

\end{vita}
 % Do not include for a candidacy proposal

%************************************************
%************************************************

% Make the Table of Contents and lists
\tableofcontents %Table of contents
\listoftables %List of tables (automatic)
\listoffigures %List of figures (automatic)
\begin{listofsymbols}
	 $\psi$ \dotfill Angular Flux

$\phi$ \dotfill Scalar Flux %List of symbols (entered manually in
	 % symbols.tex)
 \end{listofsymbols}

%************************************************
%************************************************

 % The following is a list of chapters.  Each is brought in from a
 % separate file using the \include{} command.

\newpage
\chapter{Introduction}\label{chap_intro}

The main idea behind this dissertation is to provide an alternative, improved solution for unfolding neutron spectra from given detector responses for the purpose of determining more accurate dose information. This chapter will outline the motivation for this problem.

\section{Background and Motivation}\label{background}
Radiation is prevalent in our lives, whether it originates from background radiation, cosmic rays, medical processes, or nuclear reactors. Understanding the effects of radiation is crucial for the ability to utilize it as well as ensure the least harm comes from it. The study of health physics is concerned with, among other things, the physical measurements of radiation and the relationship that arises between radiation exposure and biological damage \cite{cember2017}. Additionally, the effects of radiation on non-biological materials is another major area of study \textbf{(need citation)}. To determine the effects of any kind of radiation that has interacted with a material, the energy of that radiation must be known \cite{cember2017}. 

The unit of radiation energy is the electron volt (eV) \cite{cember2017}. This unit is derived from the energy required to move an electron through one volt of potential difference. The electron volt is equivalent to $1.6 \cdot 10^{-19}$ Joules. Typically, radiation can have a range of energy from meV ($10^{-3} eV$) to GeV ($10^9 eV$). When quantifying the effects of radiation interacting with materials, the type of radiation must also be known \cite{knoll_2020}. Due to their unique types of interactions, radiation is categorized into four categories: heavy charged particles, fast electrons, electromagnetic radiation, and neutrons \cite{knoll_2020}. In the study of radiation, the energy range of $10 eV$ to $20 MeV$ is of most interest \cite{knoll_2020} (p1). The lower limit of $10 eV$ is chosen because that is the transition point above which radiation becomes ionizing \cite{knoll_2020} (p1). Because of their importance in many nuclear reactions, thermal neutrons, which have energies as low as $40 meV$, are an exception \cite{knoll_2020} (p1).

Depending on the category and energy of the radiation, the types and effects of interaction vary widely \cite{evans}.

The ability to accurately measure and calculate radiation dose to within certain limits is a standard set my national and international government organizations \cite{doe-std-1098}. On a regular basis, the capabilities of nuclear enterprises are tested against these standards through multi-organizational exercises \textbf{cite: either DOE O 420.1C, 10 CFR Part 830, or IER-538 CED4A Report}. The purpose of these exercises is to acquire detector responses through dosimeters that can be used to obtain dose information \textbf{10 CFR Part 830}. Acquiring dose information requires multiple steps from detector response to dose. First, the detector response needs to be unfolding to show the neutron spectrum. This process is not well defined and is currently under study to improve it. Second, that unfolded neutron spectrum must be converted to dose through reported energy-dependent conversion factors \cite{compendium_of_neutron_spectra}.

The study of neutron spectrum unfolding can be traced back to the 1950s \cite{poole1952} with further work on developing specific neutron spectrum unfolding techniques in the 1960s \cite{habiger1964}. The accurate unfolding and characterization of neutron spectra from detector responses in important in many fields of study and in practice. In the last 60-70 years, various methods have been developed to improve the accuracy and reliability of spectrum unfolding, including but not limited to: gathering information from proton recoil \cite{thomas1999}, using nuclear reaction products \cite{weyrauch1993}, time-of-flight methods \cite{tagliente1998}, threshold methods \cite{hecker1977}, and multiple sphere systems \cite{bonner1960}.

\section{Dissertation outline}\label{sec_dissertation_outline}
It goes like this. %Introduction Chapter

\newpage
\chapter{This is Another Chapter}\label{chap_chap2}

You can use any equation environment.
I suggest the ``align" environment.
Like this:
\begin{align}\label{eq1}
1+1=2 \,.
\end{align}
This is because this environment allows you to break your equations easily.
If an equation is broken, always number its first line only:
\begin{align}\label{eq2}
(x+y+z+a +b &+c+d+e+f+g+h) + (w+r+t+y+u) = \\
& (x+y+z+a+b+c+d+e+f+g+h+w+r+t+y+u) \nonumber
\end{align}

You can also also use the ``subequations" environment.
Identify a list of subequations by [pluraleq], like this: 
\begin{subequations}\label[pluraleq]{eq3}
\begin{align}
1+2 &= 3\,, \label{eq3_a}\\
2+3+5 &= 10\,,\label{eq3_b}\\
1+2+2+3+5 &= 3+10\,.\label{eq3_c}
\end{align}
You can even add text between the list of equations.
\begin{align}\label{eq3_d}
13 = 13\,.
\end{align}
\end{subequations}

Now you can use the ``cleverref" package to refer to equations, tables, chapters, sections, figures, etc.
Just enter \cref{eq1} or \Cref{eq1} if you do not want abbreviations.
Also, the list of equations is done automatically: \cref{eq3}. 
Or you can do \cref{eq1,eq3_b,eq3_c,eq3_d}. 

The same package will refer to figures and tables, and these will be entered automatically in your index.

\section{Figures and Tables}\label{sec_21}

This is an example of a basic table:
\begin{table}[htb]
\centering
\caption{Example of a table}\label{tab_1} 
\begin{tabular}{||c|c|c|c|c||c|c|c|c|c||} \hline \hline
\textbf{Set}  & $\ell_1$ & $\ell_2$ & $\Sigma_{t1}$ &$q_1$ & \textbf{Set}  & $\ell_1$ & $\ell_2$ & $\Sigma_{t1}$ &$q_1$ \\ \hline\hline
$A_1$ & 0.5 & 1.0 & 1.0 & 1.0 & $B_1$ & 20/3 & 40/3 & 1.5 & 1.5\\
\hline
$A_2$ & 1.0 & 1.0 & 1.0 & 1.0 & $B_2$ & 10 & 10 & 1.0 & 1.0\\
\hline
$A_3$ & 1.0 & 0.5 & 1.0 & 1.0 & $B_3$ & 40/3 & 20/3 & 0.75 & 0.75\\
 \hline\hline  
  \end{tabular}
\end{table}

This is an example of a basic figure:
\begin{figure}[htb]
  \centering
  \includegraphics[scale=0.3]{images/fig1}
  \caption{Example of a figure} \label{fig_1}
\end{figure}
As you can see, ``cleverref" allows you to refer to more than just equations: you can refer to \cref{chap_chap2}, or \cref{sec_21}, or \cref{tab_1}, or \cref{fig_1}.
Tables and figures are indexed automatically (see index).

 % Chapter 2

\newpage
\chapter{Machine Learning/Neural Networks}\label{chap_chap3}
This Chapter will contain all of the current machine learning and neural networks techniques.

\section{Input Normalization}\label{Input_Normalization}

\begin{itemize}
\item The benefit of normalizing input data is that it will drastically reduce calculation time and it is very important to get good results. The data used in this paper has varying units and a vast range of values (ranging for multiple orders of magnitude). Large values used in neural networks use up much more storage space and significantly slow down computations. They used different normalization techniques (which were sometimes just a linear transformation rather than a normalization) to test the effects on the learning process. \ref{Input_Normalization}
\item Normalization is critical to getting the data in a readable and easily interpreted format. Without normalization, the network-based model may not be able to find a positive correlations between the variables. Almost all data in scientific studies will have been normalized before it is used. The normalization of the data scales it to the same range, which reduces analysis time and minimizes the bias in the ANN. This paper focuses on four normalization methods \ref{Norm_In_Sample_Size}
\begin{itemize}
\item Z-score Method (which is actually the same method that SciKit calls StandardScaler): $x=\frac{x-\mu}{\sigma}$
\item Min-Max Method (SciKit\_MinMaxScaler): $x = \frac{x_i-x_{min}}{x_{max}-x_{min}}$
\item Median Method: $x = \frac{x_i}{Median} $
\item Adjusted Min-Max Method $x = 0.8*\frac{x_i-x_{min}}{x_{max}-x_{min}} + 0.1$
\end{itemize}
The concluded that the adjusted Min-Max Method was the best for their dataset, but they do not make any claims about it being better in general.

\end{itemize}

\section{jude}

 % Chapter 3

\newpage
\chapter{Unfolding Neutron Spectra with Neural Networks}\label{chap_chap4}
This Chapter will contain all of the process of developing a neural network to unfold neutron spectra.

\newpage
\chapter{Validation of Unfolding Neutron Spectrum Using a Neural Network}\label{chap_chap5}
This Chapter will contain all of the work done validating the neural network for unfolding spectra.

\newpage
\chapter{Comparing Neutron Spectrum Unfolding Techniques}\label{chap_chap6}
This Chapter will compare the neural network unfolding method with other unfolding techniques.

\newpage
\chapter{Unfolding the Spectrum with New Data}\label{chap_chap7}
This Chapter will showcase the results of using the neural network to unfold data from a new real world detection using the PNS.

\newpage

\chapter{Conclusions}\label{chap_conc}

This is your final chapter. 
It does not have to be titled ``Conclusions"; it could be ``Discussion", or whatever else you prefer.

I am going to use this chapter to talk about references.
All your references should be in the references.bib file, in the same folder as this source file.
Only entries that are actually referenced in the text will show up, so you do not have to delete entries from the references.bib file.
References will appear in the order they are cited in the text.
You can look at the references.bib file to see how to enter each of the references in the following examples.

For papers in proceedings you need: authors' names, title of paper, title of proceedings, location [city, state (if in the US), or city, country (if abroad)], dates (month, days, year).
See examples in \cite{proc1,proc2,proc3}.

For papers in journals you need: authors' names, title of paper, full name of journal, volume, number (if exists), pages, year.
See examples in \cite{artic1,artic2,artic3}.

For book chapters or papers in books you need: authors' names, title of chapter or paper, title of book, name of editors, name of publisher, pages, year.
See examples in \cite{chapter1,chapter2,chapter3}.

For books you need: authors' names, title of book, name of publisher, year.
See examples in \cite{book1,book2,book3}.  % Conclusion Stuff

%************************************************
%************************************************

    %
    % If you have appendices in your dissertation, you will need the
    % following, else keep it commented. The following appendices should be in
    % files called ``app1.tex'' and ``app2.tex'', and they
    % look just like any chapter. You may rename it as you want.
    %

\appendix
\chapter{This is an Appendix}\label{app1}

You can have as many appendices as needed.
%\include{app2}

%************************************************
%************************************************


%
% The all important bibliography file at the end of your document!! 
% I have already set the preferred style; do not change it unless necessary.
% Examples of how to use them are given in the ``conclusion" chapter
% of this document.

\printbibliography

\end{document}


